\documentclass[a4paper,oneside]{report}
\usepackage[spanish]{babel}
\usepackage[latin1]{inputenc}
\usepackage{fullpage}
\usepackage{listings}
\usepackage{fancyvrb}
\usepackage[colorlinks=true,urlcolor=black,linkcolor=black]{hyperref}%

\setlength{\parskip}{1ex plus 0.5ex minus 0.2ex}

\lstset{language=,keywordstyle=\ttfamily,stringstyle=\ttfamily}
\lstset{breaklines}

title{Compiladores e Int�rpretes\\Informe del Segundo Proyecto

\author{Diego Marcovecchio (LU: 83815)\and Leonardo Molas (LU: 82498)}

\date{26 de Octubre de 2010}

\begin{document}
	
\maketitle
		
\tableofcontents

\chapter*{Introducci�n}
\section*{Descripci�n}

Este proyecto consiste en la implementaci�n de un int�rprete de un subconjunto del lenguaje funcional ML, que devolver� como salida el tipo de la expresi�n ingresada. Para esto se utilizaron herramientas autom�ticas (JLex y CUP) 

\chapter{Modo de uso}

\section{Requerimientos}

\section{Ejecuci�n}

\section{Formato de salida}

\chapter{Lenguaje}
\section{Alfabeto de entrada}


\section{Errores detectados}

\chapter{Analizador l�xico}
\section{Palabras reservadas}


\end{document}
